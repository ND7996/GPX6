\documentclass{article}
\usepackage{graphicx} % Required for inserting images
\usepackage[numbers]{natbib} % For numerical citations

\title{Computational analysis of the evolution of glutathione peroxidase 6 (GPX6) activation free energy}
\author{Nayanika Das \\ Department of Biosciences, Universitat de Vic - Universitat Central de Catalunya, 08500 Vic, Spain}
\date{\today}

\begin{document}
\maketitle

\section{Methodology}

\subsection{Empirical Valence Bond (EVB)Implementation}

\subsubsection{Free Energy Calculation Using Qfep}

Performing free energy perturbation (FEP) calculations with Q involves running a set of consecutive input files which have the mapping parameter $\lambda$ ranging in a desired way (usually [1, 0] to [0, 1] for two states). Qfep is a program which reads the energy files generated by Qdyn and calculates the total change in free energy for the complete perturbation from state A ($\varepsilon_1$) to state B ($\varepsilon_2$). Zwanzig’s formula calculates the difference in free energy between the two states:

\begin{equation}
    \Delta G = \sum \Delta g = \sum -R \cdot T \cdot \ln \left\langle e^{-\left(\frac{\Delta V_{\text{eff}}}{R \cdot T}\right)} \right\rangle_{A}
\end{equation}

Here, \(\Delta V_{\text{eff}}\) is defined as the difference in \( V_{\text{eff}} \) between two adjacent perturbation steps. 

The program returns a list containing average energies and \(\lambda\) values for each file. After that, the free energy change between each perturbation step (file) is summarized. The change is calculated relative to both the previous and the following perturbation step (\(dG_f\) and \(dG_r\) for forward and reverse directions, respectively). The accumulated sum of the energy changes between \(\varepsilon_1\) to \(\varepsilon_2\) is also given (\(\text{sum}(dG_f)\) and \(\text{sum}(dG_r)\)), as well as the average accumulated change calculated from the forward and reverse directions (\(h dG_i\)).

Qfep also calculates free energy functions, or potentials of mean force, using the perturbation formula. The reaction coordinate \(X\) is defined as the energy gap between the states \(X = \Delta V = \varepsilon_1 - \varepsilon_2\) and is divided into intervals \(X_m\) (bins). The first term in the equation represents the free energy difference between the initial state \(\varepsilon_1\) and the mapping potential \(V_i\):

\begin{equation}
    \Delta G(X_m) = \Delta G (\lambda_i) - R \cdot T \cdot \ln \left\langle e^{-\left(\frac{E_g(X_m) - V_i(X_m)}{R \cdot T}\right)} \right\rangle_{i}
\end{equation}

The second term represents the free energy difference between the mapping potential \(V_i\) and the ground state potential \(E_g\). The MD average in this term is only taken over those configurations where \(X\) belongs to \(X_m\).

\begin{equation}
    \Delta G(\lambda_i) = - R \cdot T \cdot \ln \left \sum_{n=0}^{i-1} \left\langle e^{-\left(\frac{V_{n+1} - V_n}{R \cdot T}\right)} \right\rangle_{n} \right
\end{equation}


\(E_g\) is the solution to the secular determinant. The system is then represented by a n × n EVB Hamiltonian.

\[ 
  \left[ {\begin{array}{cc}
    H_{11} & H_{12} \\
    H_{21} & H_{22} \\
  \end{array} } \right]
\]

Here, \(H_{11}\) and \(H_{22}\) are the energies of the two valence states which are calculated using classical force fields. 
For a two-state representation, the solution becomes:

\begin{equation}
    E_g = \frac{1}{2} \cdot \left( \epsilon_1 + \epsilon_2 \right) - \frac{1}{2} \sqrt{ \left( \epsilon_1 - \epsilon_2 \right)^2 + 4 \cdot H_{12}^2 }
\end{equation}

where \(H_{ij}\) or \(H_{12}\) is the off-diagonal matrix element representing the quantum mechanical coupling of the states. This coupling is zero for normal FEP calculations. \(H_{ij} \neq 0\) results in the mixing of states \(i\) and \(j\), which is desired when calculating reaction free energy profiles for reactions represented by the empirical valence bond (EVB) model. In Qfep, the off-diagonal element is a function of the form:

\begin{equation}
    H_{ij} = A_{ij} \cdot (e^{-(\mu (r_{ij} - r_0) + \eta (r_{ij} - r_0)^2)})
\end{equation}

where \(r_{ij}\) is the measured distance between the reacting atoms. By choosing \(\mu\) and \(\eta\) differently, \(H_{ij}\) can be either a constant, an exponential function, or a Gaussian function. The EVB method allows calibration of simulated reference reactions to experimental data obtained from gas-phase or solution experiments. The two EVB parameters \(H_{ij}\) (mostly \(A_{ij}\)) and \(\Delta \alpha_{ij}\) are varied until the calculated profile and the experimental data coincide. \(\Delta \alpha_{ij}\) is a constant energy shift between the states that represent their difference in heat of formation, which is not included in the force field. Generalized, the \(\Delta \alpha_{ij}\) parameter determines the \(\Delta G^\circ\) level, and \(H_{ij}\) regulates the degree of mixing of the states at the transition state, i.e., the \(\Delta G^\ddagger\) level.

The energies describing the FEP functions and the reaction free energy profile are summarized in the last table generated by Qfep. Note that each bin has contributions from several different values of \(\lambda\). Likewise, each value of \(\lambda\) contributes to the sampling of several different bins.
