\section{Results and Discussion}

\subsection{GSSG Binding to GPX6}

\subsection{Reaction Mechanisms}

Figure \ref{fig:mechanism} shows essentials of the DFT-calculated oxidation of SecGPx by $H_{2} O_{2}$. The calculation was performed with E, as shown in Fig. 2, but with a single water molecule and $H_{2} O_{2}$ bound to the reaction center. The polar contact network of E is shown as yellow dashed lines. The tetrad residues Sec46, Asn137, Trp136, and Gln81, as well as Gly47 contribute to catalysis. The transition state, leading to the charge-separated form CS, involves three concurrent steps: i) proton transfer from Sec to hydrogen peroxide, ii) proton transfer from hydrogen peroxide to water molecule, iii) proton transfer from water to nitrogen of the tryptophan ring. CS is bearing a negative charge on selenium and a positive one on the ring nitrogen of tryptophan. CS evolves with no energy barrier to the selenenic form F and a water molecules via a concerted mechanism. Formation of the chargeseparated form does not depend on binding of $H_{2} O_{2}$ but just on the presence of water. Binding of $H_{2} O_{2}$ to CS then generates the identical unstable CS•$H_{2} O_{2}$ complex which decays as described. The E to F transition is initiated by formation of a weak complex with $H_{2} O_{2}$ ([E•$H_{2} O_{2}$] in Fig. 1 and 2). $H_{2} O_{2}$ is fixed between the carbonyl oxygen of the Asn137-Phe138 bond, the selenol and the water, which in turn is integrated in a relay of mobile protons spanning from the Asn137 carboxamide nitrogen over the Gln81 carboxamide and the water to the imino nitrogen of Trp136. In the transition state (TS1), the selenol proton moves via $H_{2} O_{2}$ and water to the Trp nitrogen, thus creating a charge-separated species. The resulting complex [CS•$H_{2} O_{2}$] is so unstable that it decays without any energy barrier.\cite{orian_selenocysteine_2015}




\subsection{Free Energy Profiles for Canonical Mouse and Human GPX6}

\subsection{Free Energy Profiles for Ancestral GPX6}