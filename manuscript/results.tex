\documentclass{article}
\usepackage{amsmath}
\usepackage{graphicx}

\begin{document}

\title{Your Title Here}
\author{Your Name}
\date{\today}
\maketitle

\section{Results and Discussion}

Our comprehensive study on the wild type and variants demonstrates the evolution of glutathione peroxidase protein 6 cys and selenium protein. 
Despite the differences in the cysteine and selenocysteine parameters in the system, they show a wide range of associated evolutionary pathways.
We first optimized the reaction, one step in a stepwise mechanism going from selenol to selenolate of selenocysteine and from thiol to thiolate ion in the case of cysteine, while setting up the Empirical Valence Bond Simulation.
The substrate hydrogen peroxide was placed at the distance from selenocysteine/cysteine and glutamine83 as mentioned in the QM/MM calculations by Morokuma.

\subsection{Free Energy Changes}

Here, we present the free energy changes for the wild type and mutant systems. The table below summarizes the mean free energy values (Mean ΔG* and Mean ΔG0) for the systems we analyzed.

% Include the external table
\documentclass{article}
\usepackage{amsmath}
\begin{document}
\begin{table}[ht]
    \centering
    \begin{tabular}{|c|c|c|}
    \hline
    Sample & Mean dG* (kcal/mol) & Mean dG0 (kcal/mol) \\
    \hline
    \hline
    \end{tabular}
    \caption{Free Energy Changes}
\end{table}
\end{document}


\end{document}
