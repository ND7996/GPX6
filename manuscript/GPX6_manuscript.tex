\documentclass[journal=jacsat,manuscript=article]{achemso}

% Image-related packages
\usepackage{graphicx}
\usepackage{subcaption}
\usepackage[export]{adjustbox}
\usepackage{wrapfig}

\newcommand*\mycommand[1]{\texttt{\emph{#1}}}

\author{Nayanika Das}
\affiliation[UVicUCC]{Computational Biochemistry and Biophysics Lab, Research Group on Bioinformatics and Bioimaging (BI$^2$), Department of Biosciences, Universitat de Vic - Universitat Central de Catalunya, 08500 Vic, Spain}
\author{Gunjan Paul}
\affiliation[UAB]{Husband} 
\author{Vijay Baladhye}
\affiliation[SPPU]{Savitribai Phule Pune University, Pune, India}  % Fixed typo "Punr" to "Pune"
\author{Jordi Villà-Freixa}
\email{jordi.villa@uvic.cat}
\affiliation[UVicUCC]{Computational Biochemistry and Biophysics Lab, Research Group on Bioinformatics and Bioimaging (BI$^2$), Department of Biosciences, Universitat de Vic - Universitat Central de Catalunya, 08500 Vic, Spain}
\alsoaffiliation{IRIS-CC}

\title[Computational analysis of GPX6 activation free energy]
  {Computational analysis of the evolution of glutathione peroxidase 6 (GPX6) activation free energy}

\abbreviations{IR,NMR,UV}
\keywords{Ancestral enzyme reconstruction, enzyme design, empirical valence bond, free energy calculations}

\begin{document}

\begin{abstract}
Outstanding success in computational protein design has been achieved in recent years by combining machine learning approaches with physicochemical properties analysis of the explored variants. Despite these efforts, however, less success has been obtained in designing good computational protocols for optimization of enzyme activity. Here we propose the use of the Empirical Valence Bond method to evaluate free energies of activation of enzyme variants to obtain reasonable mutational pathways leading to a functionally optimized protein. In particular, we propose a method that explores a lower free energy difference pathway for the directed evolution of enzymes based on EVB activation free energies. To test the idea, we study the hypothetical connectivity of the selenocysteine-containing human glutathione peroxidase 6 protein (GPX6) into its ortholog cysteine-containing mouse GPX6. We show how it is possible to find a mutational pathway connecting the two protein sequences and structures that provide the empirical barriers for the first step of the GPX6-catalyzed reaction. Moreover, this protocol offers the potential for addressing complex issues such as epistasis in enzyme engineering, further enhancing its utility in enzyme optimization.
\end{abstract}

\end{document}
