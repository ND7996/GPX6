\documentclass[journal=jacsat,manuscript=article]{achemso}

% Image-related packages
\usepackage{graphicx}
\usepackage{subcaption}
\usepackage[export]{adjustbox}
\usepackage{wrapfig}
\usepackage[version=4]{mhchem}
\usepackage[numbers, square]{natbib} % Use natbib for citations

\newcommand*\mycommand[1]{\texttt{\emph{#1}}}

\author{Nayanika Das}
\affiliation[UVicUCC]{Computational Biochemistry and Biophysics Lab, Research Group on Bioinformatics and Bioimaging (BI$^2$), Department of Biosciences, Universitat de Vic - Universitat Central de Catalunya, 08500 Vic, Spain}
\author{Gunjan Paul}
\affiliation[UAB]{University Autonomous de Barcelona}
\author{Vijay Baladhye}
\affiliation[SPPU]{Savitribai Phule Pune University, Pune, India}
\author{Jordi Vill\`a-Freixa}
\email{jordi.villa@uvic.cat}
\affiliation[UVicUCC]{Computational Biochemistry and Biophysics Lab, Research Group on Bioinformatics and Bioimaging (BI$^2$), Department of Biosciences, Universitat de Vic - Universitat Central de Catalunya, 08500 Vic, Spain}
\alsoaffiliation{IRIS-CC}

\title[Computational Analysis of GPX6 Activation Free Energy]
  {Computational Analysis of the Evolution of Glutathione Peroxidase 6 (GPX6) Activation Free Energy}

\abbreviations{IR,NMR,UV}
\keywords{Ancestral enzyme reconstruction, enzyme design, empirical valence bond, free energy calculations}

\begin{document}

\maketitle

\begin{abstract}
Outstanding success in computational protein design has been achieved in recent years by combining machine learning approaches with physicochemical property analysis of protein variants. Despite these efforts, optimization of enzyme activity remains challenging. Here, we propose a method combining the Empirical Valence Bond (EVB) method with computational free energy calculations to evaluate mutational pathways leading to optimized enzyme activity. Using GPX6 as a model, we compute activation free energy differences between the human (selenocysteine-containing) and mouse (cysteine-containing) orthologs. Our results provide insights into the evolutionary dynamics of selenium usage in enzymes.
\end{abstract}

\section{Introduction} \label{sec:intro}

Selenium (Se), in the form of selenocysteine (Sec, U), the 21st amino acid, is found in 25 human proteins. Sec insertion into proteins involves recoding the UGA stop codon as a sense codon, a process requiring unique biological machinery \\cite{Hondal2011}. The chemical advantages of Sec over cysteine (Cys) include its enhanced nucleophilic character and a much lower pKa, making it more suitable for redox reactions \\cite{Hondal2011,Cardey2007}. These properties are critical in enzymes such as glutathione peroxidases (GPXs), which protect cells from oxidative stress by catalyzing the reduction of peroxides.

\subsection{Mechanism of GPX}

The catalytic mechanism of GPX has been extensively studied. A mechanism proposed for GPX3 by Prabhakar et al., using DFT calculations, involves the selenol form of Sec reacting with hydrogen peroxide, with a computed activation barrier of 16.4 kcal/mol \\cite{Prabhakar2006}. Other studies suggest alternative pathways, such as proton transfer via water or direct reduction by the selenolate form \\cite{Orian2015, Flohe2022}. These studies highlight the flexibility and efficiency of Sec in catalysis.

\section{Empirical Valence Bond Model}

The EVB model offers a computationally efficient alternative to quantum mechanics/molecular mechanics (QM/MM) approaches for simulating enzymatic reactions. The EVB potential energy is defined as:
\begin{equation}
    E_{EVB} = E_{0} + \sum_{i=1}^{N} \frac{1}{2} k_i \Delta x_i^2 + V_{0},
\end{equation}
where \(\Delta x_i\) represents deviations of reaction coordinates, \(E_0\) is the system's equilibrium energy, and \(V_0\) is a constant. This approach captures environmental effects on reaction energetics, enabling efficient analysis of enzyme evolution and mutational pathways \\cite{Carvalho2014}.

\section{Results and Discussion}

Using EVB, we calculated the activation free energy for the GPX6-catalyzed reaction in both human (Sec) and mouse (Cys) variants. Our results show significant differences in activation barriers, highlighting the role of Sec in optimizing enzyme function. Additionally, mutational pathways derived from our simulations suggest evolutionary strategies that preserve catalytic efficiency while adapting to environmental pressures.

\section{Conclusions}

This study demonstrates the utility of EVB modeling in exploring enzyme evolution and optimizing catalytic activity. By analyzing the activation free energy of GPX6 variants, we provide insights into the functional advantages of selenium-based catalysis and propose computational strategies for enzyme engineering.

\bibliography{/home/hp/nayanika/github/GPX6/manuscript/references}
\bibliographystyle{achemso}

\end{document}
